\documentclass{article}
\usepackage[utf8]{inputenc}

\title{2 Page Aricle}
\author{Madison Lytle}
\date{February 2021}

\begin{document}
% Include a two-page article of your results. Your article should be appropriate for inclusion in an introductory college level biology textbook to discuss recent developments in our understanding of the roles fungi play in ecological systems.

Natural ecological systems rely on an expanse of cycling, from carbon to energy. Materials pass in and out of organisms and throughout environments, each piece occupying niches that play roles in determining the total structure.

Fungi, a group of eukaryotic organisms \footnote{"Eukaryotes are unicellular or multicellular organisms, which have membrane enclosed organelles such as specially nucleus, mitochondria, golgi apparatus and chloroplasts in plants". \cite{Lakna2017}}, are one such piece that p\hold an integral role in this continuous system. They serve as the interface between other living organic material and the soil with a important job: decomposition.

Decomposition is the process of breaking down larger substances, called substrates, into smaller molecules that can be realised into the soil or air. Without decomposers like fungi, the nutrient cycle would be halted. This role has been long established \cite{Kakde2009}. However, more precise mechanisms and models for fungal decomposition are being progressively developed. In particular, we have built a greater understanding of more specialized fungal niches and response to environmental changes on various time scales.

Within the role of decomposer, there are yet more divisions of work. Sepcific species of fungi are abel to sucrete different enzymes capable of breaking down distinct substrates. Alone, a single species of fungi would not be able to break down a log, but a community of fungi will. Amongst a variety of specific fungal roles, there emerges a pattern of two factions of fungal types. %need better word than factions

The first type if that of the competitive, fast growing fungi. In the right conditions, these fungi grow and decompose material faster than the other grouping. This fast growth is indicated by the hyphal extension rate. Fungi are composed of long interwoven strands of filament, know as hyphae, that grow by extending outwards from Spitzenkorper - a vesicle bundle located at the tip of each strand \cite{Edelstein1982}. Given this growth nature, hyphal extension rate can be used as a proxy measure for overall fungal growth. These species are opportunists, meaning they seek out specific conditions in which to seize the opportunity to grow. Indeed, these species are particularly environmentally dependent. Despite generally having these higher hyphal extension rates, in order to reach there competitive and productive potential they require more optimal growth conditions. For example, if the species are place in an environment with limited access to water, meaning there is low moisture potential, often there advantage over the other species is little to none.

[FIG]
\caption{Hyphal extension rate, $\nu$, compared with resulting average contribution to decomposition for 35 different species of fungi studied in Maynard et al. 2019 in 5 different sample biomes with varying temperatures and soil moisture potentials, demonstrating the importance of of environmental conditions to more competitive species.}

By comparison, the advantage of the second group of fungi relies more on their resilience in a variety of environmental conditions rather than in there speed of growth. When placed in environments with more or less optimal moisture and temperature, they maintain a roughly consistent growth rate, increasing growth to more favorable conditions at a fraction of the increase demonstrated by fast-growing fungi.

Both niches are significant to a fluctuating environment. Having a biodiversity of fungal types supports the entire niches ability to break down organics matter, with opportunists making for quicker decomposition as possible and slower-growing fungi serving as a steady, more-permanent  source.

In addition to our understanding of this further niche differentiation, we have also gained insight into how these environmental fluctuations themselves may impact decomposition. Changes on short time scales, such as seasonal variations of temperature and moisture availability, have potentially impacted the system by both creating incentive for a diverse array of fungal species. When this variation was implemented, it resulted in increased growth rates.

On larger time scales, climate and geological changes care result in evolving biomes. This can be thought of as a common general environmental configuration transforming into another, such a a temperate forest transitioning into and arid dessert or a tundra transitioning into a grassland as global temperatures increase. In general, of environments examined, fungal species tended to grow quicker in environments with higher water potentials. For example, a sampled tropical rain forest contained significantly faster growth rates than other sampled biomes. This agrees with the rain forests' relatively increased rate of nutrient cycling, with high average plant growth \cite{Lodge1993}.

Interactions between members of ecological systems are complex and intricate: there are many mechanisms of fungal decomposition yet to be understood and investigated. However, our understanding of fungi and their roles will continued to be studies, as more advanced techniques for both modeling interactions both experimentally and mathematically.


\end{document}


to cite:

https://www.researchgate.net/publication/314051827_Difference_Between_Prokaryotic_and_Eukaryotic_Cells

https://www.researchgate.net/publication/297739615_STUDIES_ON_FUNGI_RESPONSIBLE_FOR_BIODEGRADATION_AND_HUMIFICATION_OF_ORGANIC_MATTER

https://www.researchgate.net/publication/240642806_Nutrient_cycling_by_fungi_in_wet_tropical_forests
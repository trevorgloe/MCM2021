%%%%%%%%%%%%%%%%%%%%%%%%%%%%%%%%%%%%%%%%
%% MCM/ICM LaTeX Template %%
%% 2021 MCM/ICM           %%
%%%%%%%%%%%%%%%%%%%%%%%%%%%%%%%%%%%%%%%%
\documentclass[12pt]{article}
\usepackage{geometry}
\geometry{left=1in,right=0.75in,top=1in,bottom=1in}

%%%%%%%%%%%%%%%%%%%%%%%%%%%%%%%%%%%%%%%%
% Replace ABCDEF in the next line with your chosen problem
% and replace 1111111 with your Team Control Number
\newcommand{\Problem}{MCM Problem A}
\newcommand{\Team}{2125756}
%%%%%%%%%%%%%%%%%%%%%%%%%%%%%%%%%%%%%%%%

\usepackage{newtxtext}
\usepackage{amsmath,amssymb,amsthm}
\usepackage{newtxmath} % must come after amsXXX

\usepackage[pdftex]{graphicx}
\usepackage{xcolor}
\usepackage{fancyhdr}
\usepackage{lipsum}
\usepackage{multicol}

\usepackage[utf8]{inputenc}
\usepackage[english]{babel}

\usepackage[
backend=biber,
style=numeric,
sorting=ynt]{biblatex}
\addbibresource{masterbib.bib}

\lhead{Team \Team}
\rhead{}
\cfoot{}

\newtheorem{theorem}{Theorem}
\newtheorem{corollary}[theorem]{Corollary}
\newtheorem{lemma}[theorem]{Lemma}
\newtheorem{definition}{Definition}

%%%%%%%%%%%%%%%%%%%%%%%%%%%%%%%%
\begin{document}
\graphicspath{{.}}  % Place your graphic files in the same directory as your main document
\DeclareGraphicsExtensions{.pdf, .jpg, .tif, .png}
\thispagestyle{empty}
\vspace*{-16ex}
\centerline{\begin{tabular}{*3{c}}
	\parbox[t]{0.3\linewidth}{\begin{center}\textbf{Problem Chosen}\\ \Large \textcolor{red}{\Problem}\end{center}}
	& \parbox[t]{0.3\linewidth}{\begin{center}\textbf{2021\\ MCM/ICM\\ Summary Sheet}\end{center}}
	& \parbox[t]{0.3\linewidth}{\begin{center}\textbf{Team Control Number}\\ \Large \textcolor{red}{\Team}\end{center}}	\\
	\hline
\end{tabular}}
%%%%%%%%%%% Begin Summary %%%%%%%%%%%
% Enter your summary here replacing the (red) text
% Replace the text from here ...
\begin{center}
\textcolor{red}{%
\lipsum[1]
}
\end{center}
% to here
%%%%%%%%%%% End Summary %%%%%%%%%%%

%%%%%%%%%%%%%%%%%%%%%%%%%%%%%%
\clearpage
\pagestyle{fancy}
% Uncomment the next line to generate a Table of Contents
%\tableofcontents 
\newpage
\setcounter{page}{1}
\rhead{Page \thepage\ }
%%%%%%%%%%%%%%%%%%%%%%%%%%%%%%


\begin{multicols}{2}
	
%\section{Pre-planning analysis}
%- 
	
\section{Introduction}

Decomposition by fungal and microbial communities is second only to photosynthesis in driving carbon cycling in the Earth's ecological communities. \cite{McGuire2010} While this significance is rarely understated, inner mechanisms and interactions within decomposition communities have been poory understood and treated as a 'black box' in the past by scientists studying ecosystem processes. \cite{Andren1999}

Fungi are made up of hyphae: long winding filaments enmeshed together to form a thick mat of mycelium. At the frontline of battle, hyphal tips are extending forward and branching to cover new area at a velocity dictated by the hyphal extension rate, and concurrently other hyphae are dying off or merging together. \cite{Edelstein1982} Like machinery, each fungus has ideal operating conditions. Fungus native to dry or arid biomes will grow efficiently in a moisture-lacking environment when other fungus will falter. Maynard et. al. defines these operating conditions by assigning temperature and moisture niche-widths to 38 studied species of fungi. 

When fungi come into contact with each other, competitive interactions are most often observed. Tactics studied in Boddy, L. 2000 include deploying diffusable toxins, obtaining nutrients parasitically, destruction of hyphae by intentional interference, and establishing wall-like barracades to thwart attackers. \cite{Boddy2000} Most researchers, including Maynard et. al. construct a competitive ranking scale based on experimental observations to encapsulate the ramifications of these complex interactions. 

Under a constant temperature regime, previous research concludes that dominant fast-growing suppress other species which negatively impacts the overall community's efficiency, however fluctuating environmental conditions change the outcome and promote complementary growth. \cite{Toljander2006} An explanation for this is presented by Lustenhouwer et. al. suggesting a spectrum between slow-growing, stress-tolerant fungi operating in a wider environmental range than fast-growing, highly competitive fungi. \cite{Lustenhouwer2020} 

Determining which fungi are present in a community depends on the organic material, or 'substrate', being broken down. In the case of fungal-mediated wood decomposition, the substrate is made up of lignocellulose polysaccarides, and the fungi involved are equipped with specialized enzymes capable of splitting linkages in the polymers to harvest usable carbohydrates. Long-term wood decomposition follows a well-studied three-phase process: the first phase is dominated by fast and opportunistic fungus claiming easily-consumable soluble sugars, followed in the second phase by consistent and dependable decomposers that degrade the holocellulose constituting 65\%-85\% of the material. \cite{Aspergilli} Slow and highly specialized fungi trail behind in the third phase to clean up rigid lignin polymers (15\%-35\%) with powerful oxidative enzymes. \cite{Moorhead2006} This phase progression is an important agent governing community changes on the long-term scale, however, this paper will focus on only the middle phase of cellulose decomposition and will not conider lignin decomposition. Fungi containing larger proportions of enzymes like acid phosphatase which are associated with lignin decomposition are demonstrated to strongly correlate with slower decomposition rates by Lustenhouwer et. al.

An important step in modeling fungal communities is acknowledging separation between the growth rate and the decomposition rate of the fungus, which are influenced by different factors. Fungal growth models typically focus on the increase in overall biomass due to hyphal extension. As described by [Edelstein 1982], the growth of hyphae over time depends on the spacial density of hypae tips (from which hyphae can extend and grow), the tip extension rate and the hyphal death rate. The tips are created by various forms of branching. This model is summed up by the following equations:

\begin{equation} \label {eq}
\frac{\partial \rho}{\partial t} = n\nu - d
\end{equation}
\begin{equation} \label {eq}
\frac{\partial n}{\partial t} = -\frac{\partial n\nu}{\partial x} + \sigma
\end{equation}

These equations form a basis for many other models, typically expanding on the effects of nutrient concentrations in the decomposing substrate. The largest assumption made by this model that we will differ from is the assumption that the mycellium grows over an unlimited nutrient supply. Both [Edelstein 1983 Growth and Metabolism in Mycelial Fungi] and [Davidson and Park, 1997] expanded on this model to include interactions with an external substrate by considering concentrations of nutrients both inside and outside the fungi. 
\\ Another note-worthy growth model is described in [Du, Tran, Perre 2019]. This more recent model describes the growth of the fungus as a 3-dimensional reaction-diffusion equation based on microscopic growth mechanisms, also in terms of the tip density, tip extension rate, hyphal branching, and hyphal elimination rates. However this model also has an added term, the proportion of active tips. In all models, the terms for $\nu$, $\sigma$ are taken to be constants given from experimental data. As noted in [Edelstein 1982], these parameters are seen to depend on nutrient availability and environmental factors in practice. In the review of various fungal growth models described in [Bayesian model selection framework for identifying growth patterns in filamentous fungi, 2016], a specific method of modeling branching and hyphal death was found to match experimental observations in the widest variety of settings. Based on these results we will choose to implement the following relationships, comprising dichotomous branching, tip-hypha anastomosis, and hyphal death (or YHD). 
\begin{equation} \label {eq}
\sigma = \alpha_{1}*n - \mu n \rho
\end{equation}

\begin{equation} \label {eq}
d = \gamma_{1}\rho
\end{equation}

\section{Pre-planning intro}

The Five Core Tenants:
1. growth/Decomposition continuous model
- general explanation of the structure of fungi: hyphal extension rate, branching, tips, death, etc.
- proportion of enzymes relative to the phases correlates with the speed of decomposition and effectiveness in a certain phase
- substrate talk - LCI, lignin inhibition of holocellulose, common substrate compositions, how soil is different
- phases talk
- reasoning behind choosing a continuous model

2. interactions between different fungi with differing characteristics
- competitive interactions *
- synergy under richness and fluctuating temperatures *
- the more differing the characteristics of the fungi, the more likely they are to compete
- the competitive ranking *

3. sensitivity to environmental fluctuations
- slower-growth stress tolerant fungi vs fast growth more niche temp/moisture fungi determining which of these two wins out depends on how much the environment is fluctuating and what subset of fungi are present *
- fluctuating temperatures promotes growth
- moisture talk *
- biome talk

4. biodiversity's impact on systems overall efficiency
- richness promotes growth
- competition promotes efficiency *
- dominant species that 'wipe out' the community generally harm? *

5. effect on the carbon cycle
- fungi/bacteria are the only life forms which can break down lignin and efficiency break down holocellulose litter — these poysaccarides make up a huge portion of the total ground litter biomass in the earth's forests

Biodiversity's impact on the system's overall efficiency:
- the greater range you have of moisture/temperature niche widths, the more tolerant your fungal community will be overall to growing in any particular environmental because at least some constituents of the community will be close to idealized in that environment.
- competition rank correlates heavily with hyphal extension rate — fungal communities with species who are highly competitive will go one of two ways: 1. the environmental conditions are ideal for the smaller niches of the highly competitive isolates, and so those isolates will dominate out over the other ones 2. the environmental conditions are not ideal for the highly competitive isolates, so the slower but more robust isolates will win out

we have a spearman correlation coefficient relating production of each enzyme with decomposition rate — enzymes primarily involved with breaking down lignin (acid phosphatase) are strongly correlated with decreased decomposition rate.

Strong correlation that increase production of acid phospatase results in decreased competitive rank, which makes sense because on average slower-growing fungi are less competitive than faster growing ones. Competitive rank has a 0.6 spearman correlation coefficient with growth performance

Assumption: enzymes treatment of the different portions of holocellulose are the same 

Implementation: each enzyme has its production rate from the data, multiply the parameter by that production rate and sum them all together to get dC/dt

Recapitulating previous literature


Prediction:
The slower stress tolerant fungi will have a share of enzymes that indicate their slow decomposition rate, but a moisture and temperature width that indicate their stress tolerance to environmental fluctuations. 

The faster fungi will have a share of enzymes that indicate their fast decomposition rate, but a moisture and temperature width that indicates their lack of stress tolerance to environmental fluctuations.

\section{Model formulation}
Maynard et. al took data on the productive rates of 8 different enzymes (5 hydrolytic and 3 oxidative to cover the spectrum of the cellulose-degradation process) and used a standard clustering approach to determine 4 representative enzymes of the entire set: cellobiohydrolase, acid phosphatase, peroxidase and phenoloxidase. The data for production of these enzymes comes from their open-source github. 

We're making the assumption that the production of each enzyme is  equivalent to the amount of the enzyme present in the fungus. We took the production data and normalized it to z-score values. Each z-score was turned into a portion of that enzyme with $e_x$ = (z-min)/(max-min) and then we divided each portion by the sum of all enzyme portions for each enzyme $(e_x)/(e_1+e_2+e_3+e_4)$. This gives us a representation of the proportion of each enzyme relative to other enzymes. We made the assumption thatt the total mass of those four enzymes for each fungus is the same, and the fraction of enzyme mass from total mass.

We made the assumption that the enzymes are not changing over time due to C/N changes or environmental fluctuations

The time scale for our model is following the experimental results from (paper cited by question) which indicates x\% log decomposition after 3 years and y\% decomposition of logs after 5 years for a wide sampling of fungus

\subsection{Assumptions}

\begin{itemize}
	\item Growth
	\begin{itemize}
		\item[--] unlimited available resources (ground litter and woody fibers)
		\item[--] neglecting effects from direct interactions between fungi
		\item[--] PROBLEM: neglecting synergistic effects on growth/metabolic efficiency
		\item[--] non-exhaustive substrates
		\item[--] not considering N addition by anthropogenic nitrogen deposition
		\item[--] LCI < 0.7 at all times because we're focusing on wood fibers and we don't want lignin block holocellulose decay in bad ways. but we could use the formula
		\item[--] LCI is not changing over time anastomosis, the formation of reconnections. 
		\item[--] dynamics regarding the rate of nutrient absorption and transport of metabolites within the
		mycelium are ignored or simplified
		\item[--] growth dynamics incurred by the formation of concentric density rings are not considered <-- this doesn't need to exist because we're considering growth in a 1D tube
		\item[--] only considering one configuration of growth traits out of many possible options, that configuration being: tip-hypha anasmotosis, dichotomous branching, and hyphal death
		\item[--] decomposition coefficients are representative of those families of enzymes
	\end{itemize}
	\item Decomposition Kinetics
	\begin{itemize}
		\item[--] Elements can be stacked in any configuration without structural limitations
	\end{itemize}
	\item Temperature/Moisture effects
	\begin{itemize}
		\item[--] experimental data for water potentials in specific biomes/environments are representative of the conditions of those environments 
	\end{itemize}
	\item Interactions
	\begin{itemize}
		\item[--] Influences from wind are neglected
		\item[--] The drones are assumed to be unobstructed by terrain
		\item[--] The drones do not experience any malfunctions
		\item[--] The earth’s curve is neglected
	\end{itemize}
\end{itemize}


\subsection{Flight Path Sub-model}
\lipsum[3]

\subsection{Packing Sub-model}
\lipsum[4]

\subsection{Storage Location Determination}
\lipsum[5]
\[ 
L_n = \sum_{i=1}^{m} p_i(x_i, y_i) / \sum_{i=1}^{m} p_i 
\]
where $L_n$ = Location number within grouping \\
n being number of distinct locations (1 to 3) \\
i = hospitals in the group \\
p = package demand of hospital i \\
x, y = longitude, latitude of hospital i \\
The output lists a discrete set of groupings of the hospitals with the corresponding locations of the storage containers for each hospital group. In figure 1, one potential combination of hospitals grouped to specific storage locations is shown. 

% TODO: \usepackage{graphicx} required
%\begin{figure}
%	\centering
%	\includegraphics[width=0.7\linewidth]{}
%	\caption[Fig 1]{caption 0??}
%	\label{Fig 1}
%\end{figure}


\subsection{Cost Function}
I've copied the ten diff eqs from Numerical Prediction Model for Fungal Growth Coupled with Hygrothermal Transfer in Building Materials here:

hygrothermal equations:

\begin{equation}
\frac{dH}{dT}\cdot\frac{\partial T}{\partial t}=\nabla\cdot(\lambda\nabla T)+h_v\nabla\cdot(\delta_{p}\nabla(\phi p_{sat}))
\end{equation}

\begin{equation}
\frac{dw}{d\phi}\cdot\frac{\partial \phi}{\partial t}=\nabla\cdot(D_{\phi}\nabla\phi+\delta_{p}\nabla(\phi p_{sat})
\end{equation}

Chapter 15 Mathematical Modeling of Fungal Growth

\begin{equation}
\frac{\delta\rho}{\delta t}=nv-d(\rho)
\end{equation}

\begin{equation}
\frac{\delta n}{\delta t}=\frac{\delta nv}{\delta x}+\sigma(\rho, n)
\end{equation}


Where $P$ represents the total medpacks delivered, $\sum{t}$ is an estimate of the time for all the flights to occur, and $S$ represents the space left after packing all the drones (computed via the packing algorithm). Also, $C$ is our cost function output, the cost. The factor of $100000$ dividing $S$ is there to adjust the units of $S$ (it being on the order of $10^5$ while $\frac{P}{\sum{t}}$ is on the order of $10^0$) The time estimates are computed via the basic kinematic equation assuming constant speed, $\sum{t}=\sum{\frac{d}{v_d}}$ summed over all drone flights in the given plan. Here $d$ is distance traveled in a specified flight and $v_d$ is the max speed of the drone flying. This estimate of the time taken for a flight is assuming the drone is flying at max speed the whole way, and assuming equality of the time taken to fly to the hospital, and the time to fly back. The assumption of perfectly sequential ordering of the flights allows us to sum the times of individual flights to get the total time.

\section{Results and Analysis}
% TODO: \usepackage{graphicx} required
%\begin{figure*}
%	\centering
%	\includegraphics[width=0.7\linewidth]{}
%	\caption[Fig 1.]{caption 1}
%	\label{Fig 1.}
%\end{figure*}

\subsection{Model output}
\lipsum[6]

\subsection{Parameter sensitivity}
\lipsum[7]

% TODO: \usepackage{graphicx} required
%\begin{figure}
%	\centering
%	\includegraphics[width=1.1\linewidth]{}
%	\caption[Fig 2]{caption 2}
%	\label{Fig 3}
%\end{figure}

% TODO: \usepackage{graphicx} required
%\begin{figure}
%	\centering
%	\includegraphics[width=0.9\linewidth]{}
%	\caption[Fig 3]{caption 3}
%	\label{Fig 4}
%\end{figure}


\subsection{Limitations/ Further Work}
\lipsum[8]

\printbibliography
\end{multicols}
\end{document}
%%%%%%%%%%%%%%%%%%%%%%%%%%%%%%%%%%%%%%%%
%% MCM/ICM LaTeX Template %%
%% 2021 MCM/ICM           %%
%%%%%%%%%%%%%%%%%%%%%%%%%%%%%%%%%%%%%%%%
\documentclass[12pt]{article}
\usepackage{geometry}
\geometry{left=1in,right=0.75in,top=1in,bottom=1in}

%%%%%%%%%%%%%%%%%%%%%%%%%%%%%%%%%%%%%%%%
% Replace ABCDEF in the next line with your chosen problem
% and replace 1111111 with your Team Control Number
\newcommand{\Problem}{MCM Problem A}
\newcommand{\Team}{2125756}
%%%%%%%%%%%%%%%%%%%%%%%%%%%%%%%%%%%%%%%%

\usepackage{newtxtext}
\usepackage{amsmath,amssymb,amsthm}
\usepackage{newtxmath} % must come after amsXXX

\usepackage[pdftex]{graphicx}
\usepackage{xcolor}
\usepackage{fancyhdr}
\usepackage{lipsum}
\usepackage{multicol,caption}
\usepackage[tikz]{multicolrule}
\usepackage{subcaption}

\usepackage{footnote}
\usepackage{float}
\restylefloat{table}
\usepackage{adjustbox}

%\usepackage{syntonly}
%\syntaxonly


\usepackage[utf8]{inputenc}
\usepackage[english]{babel}

\usepackage[
backend=biber,
style=chem-angew,
sorting=ynt]{biblatex}
\addbibresource{masterbib.bib}

\lhead{Team \Team}
\rhead{}
\cfoot{}

\newtheorem{theorem}{Theorem}
\newtheorem{corollary}[theorem]{Corollary}
\newtheorem{lemma}[theorem]{Lemma}
\newtheorem{definition}{Definition}

%%%%%%%%%%%%%%%%%%%%%%%%%%%%%%%%

\newenvironment{ColumnFigure}
{\par\medskip\noindent\minipage{\linewidth}}
{\endminipage\par\medskip}

\begin{document}
\graphicspath{{.}}  % Place your graphic files in the same directory as your main document
\DeclareGraphicsExtensions{.pdf, .jpg, .tif, .png}
\thispagestyle{empty}
\vspace*{-16ex}
\centerline{\begin{tabular}{*3{c}}
	\parbox[t]{0.3\linewidth}{\begin{center}\textbf{Problem Chosen}\\ \Large \textcolor{red}{\Problem}\end{center}}
	& \parbox[t]{0.3\linewidth}{\begin{center}\textbf{2021\\ MCM/ICM\\ Summary Sheet}\end{center}}
	& \parbox[t]{0.3\linewidth}{\begin{center}\textbf{Team Control Number}\\ \Large \textcolor{red}{\Team}\end{center}}	\\
	\hline
\end{tabular}}
%%%%%%%%%%% Begin Summary %%%%%%%%%%%
% Enter your summary here replacing the (red) text
% Replace the text from here ...
\begin{center}
\textcolor{red}{%
In a model coupling fungal growth with standard decomposition reaction kinetics, we study the effect of varying parameters for fungal growth on the overall decomposition of a wood-based substrate by 35 species. 
Growth is simulated using a 1-dimensional growth model based on fundamental microscopic growth mechanisms, linearly coupled with the Michaelis-Menten reaction equation. Interactions between different fungi are limited to competition for a common nutrient pool. Various fungal species are differentiated by their enzymatic makeup of four different representative enzymes and their ecological nices. We use the moisture potential and temperature parameters, combined with experimental data on hyphal extension rates to represent the influence of the environment across the model. We study two areas: the relative advantages of individual fungi and the effect of biodiversity on the community decomposition efficiency. Our results support the existence of a spectrum where species either tend towards being more competitive, demonstrating a higher hyphal extension and decomposition rate, or more stress tolerant, demonstrating less variability facing long term environmental fluctuations. The introduction of yearly moisture potential fluctuations corresponds with a slight increase in overall decomposition system efficiency. The positive impact of biodiversity distribution correlates with competitive ranking coefficients up to a critical point, where further following competitive ranking coefficient distribution appears to have no correspondence with the total decomposition efficiency. The gap in understanding between short and long term fluctuations of environmental parameters and this critical point of biodiversity distribution prompt further investigation.}
\end{center}
% to here
%%%%%%%%%%% End Summary %%%%%%%%%%%

%%%%%%%%%%%%%%%%%%%%%%%%%%%%%%
\clearpage
\title{Environmental Tension and Hyphal Extension: A model for Fungi-facilitated Decomposition}
\date{}
\maketitle
\newpage
\begin{multicols}{2}
\tableofcontents
\listoftables
\listoffigures
\pagestyle{fancy}
% Uncomment the next line to generate a Table of Contents
%\tableofcontents 
\newpage
\setcounter{page}{1}
\rhead{Page \thepage\ }
%%%%%%%%%%%%%%%%%%%%%%%%%%%%%%


	
%\section{Pre-planning analysis}
%- 

\section{Introduction}

Decomposition by fungal and microbial communities is second only to photosynthesis in driving carbon cycling in the Earth's ecological communities \cite{McGuire2010}. While this significance is rarely understated, inner mechanisms and interactions within decomposition communities have been poory understood and treated as a 'black box' in the past by scientists studying ecosystem processes \cite{Andren1999}.

Fungi are made up of hyphae: long winding filaments enmeshed together to form a thick mat of mycelium. At the frontline of battle, hyphal tips are extending forward and branching to cover new area at a velocity dictated by the hyphal extension rate, and concurrently other hyphae are dying off or merging together \cite{Edelstein1982}. Like machinery, each fungus has ideal operating conditions. Fungi native to dry or arid biomes will grow efficiently in a moisture-lacking environment when other fungi will falter. Maynard et al. (2019) defines these operating conditions by assigning temperature and moisture niche-widths to 37 studied species of fungi \cite{Maynard2019}.

\subsection{Competition}
When fungi come into contact with each other, competitive interactions are most often observed. Tactics studied in Boddy, L. (2000) include deploying diffusable toxins, obtaining nutrients parasitically, destruction of hyphae by intentional interference, and establishing wall-like barracades to thwart attackers \cite{Boddy2000}. Most researchers, including Maynard et al. (2019) construct a competitive ranking scale based on experimental observations to encapsulate the ramifications of these complex interactions \cite{Maynard2019}.

Under a constant temperature regime, previous research concludes that dominant fast-growing suppress other species which negatively impacts the overall community's efficiency. However fluctuating environmental conditions change the outcome and promote complementary growth \cite{Toljander2006}. An explanation for this is presented by Lustenhouwer et al. (2020) suggesting a spectrum between slow-growing, stress-tolerant fungi operating in a wider environmental range than fast-growing, highly competitive fungi \cite{Lustenhouwer2020}. 

\subsection{Substrates}
Determining which fungi are present in a community depends on the organic material, or 'substrate', being broken down. In the case of fungal-mediated wood decomposition, the substrate is made up of lignocellulose polysaccarides, and the fungi involved are equipped with specialized enzymes capable of splitting linkages in the polymers to harvest usable carbohydrates. Long-term wood decomposition follows a well-studied three-phase process: the first phase is dominated by fast and opportunistic fungus claiming easily-consumable soluble sugars, followed in the second phase by consistent and dependable decomposers that degrade the holocellulose constituting 65\%-85\% of the material \cite{Segato2014}. Slow and highly specialized fungi trail behind in the third phase to clean up rigid lignin polymers (15\%-35\%) with powerful oxidative enzymes \cite{Moorhead2006}. While phase progression is an important agent governing community changes on the long-term scale, this paper will focus on only the middle phase of cellulose decomposition and will not conider lignin decomposition. \cite{Lustenhouwer2020}.

\subsection{Growth Model}
An important step in modeling fungal communities is acknowledging separation between the growth rate and the decomposition rate of the fungus, which are influenced by different factors. Fungal growth models typically focus on the increase in overall biomass due to hyphal extension. Fungi with a greater density of hyphal tips per unit volume will grow and accrue biomass faster, as hyphae only grow from their tips. New tips are are created when hyphae branch off, while others die or fuse together via anasmotosis. Edelstein (1982) models these dynamics with the following:

\begin{equation} \label{eq:1}
\frac{\partial \rho}{\partial t} = n\nu - d
\end{equation}
\begin{equation} \label{eq:2}
\frac{\partial n}{\partial t} = -\frac{\partial n\nu}{\partial x} + \sigma
\end{equation}

Where  $\rho$ is the total hyphal density, $n$ is the tip density, $\nu$ is the hyphal extension rate, $d$ is the hyphal death rate, and $\sigma$ is the increase of tips due to branching. These equations will operate standing alone, without any other influences by enzymes or growth limitations. This allows many models to treat it as a baseplate to expand upon. Edelstein (1983) and Davidson et al. (1997) do this by considering concentrations of nutrients inside the fungi and outside in the surrounding substrate \cite{Edelstein1983, Davidson2012}. Our model will likewise follow in eliminating the assumption that fungi grow over an unlimited nutrient supply. A more contemporary approach by Du et al. (2019) describes the growth of a fungus with a 3-dimensional reaction-diffusion equation based on microscopic growth mechanisms, also focusing on tip density and extension rate but with an added term to identify the proportion of tips which are actively growing \cite{Du2019}.

\subsection{Environmental Influence on Hyphal Tip Extension ($\nu$)}
Environmental impacts are introduced to the decomposition model through $\nu$, the hyphal tip elongation rate. The mechanism of elongation is relatively unknown; as described in Gervais et al. (1999), "cell turgor pressure corresponds to an overpressure which allows the cell morphology, elongation, division and hence the biomass evolution" \cite{Gervais1999, Steinberg2007}. In fungi, cell turgor comes from diffusion of water dictated by the water potential\footnote{Water potential ($\psi$) describes the content of availability of moisture surrounding a fungi as dictated by the soil composition.} gradients rather active transport within the cell, making it very environmentally dependent \cite{Gervais1999}. This fact justifies determining $\nu$ based on environmental factors on fungal decomposition rates by varying $\nu$ outside of baselines for given species in set locations in our model. Experimental data from Maynard et al. (2019) shows the relationship between $\nu$ and the environmental parameters of water potential ($\psi$) and temperature ($T$) \cite{Maynard2019}.

%\begin{figure*}
%	\centering
%	%\vspace*{-2ex}
%	\includegraphics[width=0.7\linewidth]{avg_contr_nu_environment.png}
%	\caption[Fig 1.]{Example optimized flight plan, ran with drone fleet size of 5, starting configuration 1, 5 allowed movements, and $\gamma$ = 2}
%	\label{Fig 1.}
%\end{figure*}
\subsection{Decomposition kinetics}
Moore et al. (2015), Edelstein (1983), Moorhead et al. (2006), Parnas (1974), and Schimel et al. (2003) describe the decomposition of a certain substrate by a microbial decomposer as a function of the substrate's concentration with the Michaelis–Menton equation \cite{Moore2015, Edelstein1983, Moorhead2006, Parnas1975, Schimel2003}. Thus, decompostion of carbon in a substrate is described by the following equation.

\begin{equation} \label{eq:3}
\frac{dC}{dt} = \frac{KBC}{K_{e}+C}
\end{equation}

Where $B$ is the biomass of fungi, $C$ is the amount of carbon in the system and $K_{e}$ is the half-saturation coefficient. Schimel et al. (2003) and Moorhead et al. (2006) assert that decomposition levels off at a maximum rate proportional to the concentration of enzymes acting on it \cite{Schimel2003, Moorhead2006}. To determine this enzyme concentration, we will make the assumption that enzyme concentration is proportional to the concentration of biomass in our system represented by $r$ multiplied by two environmental limiting factors in $K=rS_{M}S_{T}$ \cite{Schimel2003}. These environmental factors are discussed at greater length in section \ref{Other Responses to Temperature and Water Potential}.

\subsection{Enzyme interactions}

Enzymes produced by fungi are expelled into the surrounding substrate to autonomously break down macromolecules. Relative enzyme concentrations can be quantified by the production rates of each enzyme by a fungus. Production of enzymes is a metabolically taxing process, therefore fungi that produce enzymes which return much more usable energy and C than was put into making the enzyme, a high "return on investment", will be able to synthesize even more enzymes and grow rapidly \cite{Schimel2003}. Lustenhouwer et al (2020) strengthens this theory by showing through statistical analysis of 1582 various fungal isolats that fungi containing larger proportions of enzymes like acid phosphatase, which are associated with slow lignin decomposition, are demonstrated to strongly correlate with slower decomposition rates \cite{Lustenhouwer2020}. Lignin exemplifies a low return on investment, requiring powerful enzymes and greater energy input per unit C than holocellulose. But it's ultimately the fungi that have a subset of enzymes that best match their surrounding substrate which successfully grow and establish a niche. As an example, in a substrate of 70\%+ lignin, the slower growing fungi with lignin enzymes normally at a disadvantage will win out over other fungi competing for a fraction of available holocellulose.

Communities with similar fungi of similar enzymatic makeup often see a few dominant species overwhelm the rest through competition for similar resources in the substrate, but communities of fungi with greatly differing enzymes will instead form distinct niches that partition different resources in which they are seperately able to grow. In these conditions, one dominant species faces difficulty in wiping out competitors who are individually masters of their own substrates. This effect is known as niche complementarity \cite{Toljander2006}. 

Research by Toljander et al. (2006) contends that environmental fluctuations further robustify a diverse community as faster growing highly competitive fungi are hindered by conditions that lapse out of their moisture niche, and slower growing stress-tolerant fungi benefit from decreased pressure by competitors. A linear regression for the moisture niche width ($W_{mn}$) vs competitive ranking ($R_c$) across 34 different fungal species studied in Maynard et al. (2019) gives $W_{mn}= 1.84R_c + 2.9$, with an $R^2$ value of 0.227. Although the $R^2$ value of this regression is not particularly strong, an underlying negatively proportional relationship of the general data set is indicated. Maynard et al. (2019) provides backing for this niche distinction by applying principal component analysis to a variety of potential fungal niche-associated traits. The results of this analysis points towards a similar direction: fungal species will lean towards one of these two profiles \cite{Maynard2019}. 

A diverse subset of fungal niches will result in a dampened and stable community will decompose the most efficiently across a range of conditions. \cite{Toljander2006}. This conclusion has signicant implications on the continuation of carbon cycling in ecosystems, as decomposers releasing carbon into the soil and atmosphere sustain a critical role in the cycle. \cite{Lustenhouwer2020}

\section{Model Formulation}


\subsection{Growth Model}
In the review of various fungal growth models described in Lin et al. (2016), a specific method of modeling branching and hyphal death was found to match experimental observations in the widest variety of settings. Based on these results, for $\sigma$ and $d$ in (\ref{eq:1}) and (\ref{eq:2}) we will choose to implement the following relationships, comprising dichotomous branching, tip-hypha anastomosis, and hyphal death (or YHD). 
\begin{equation} 
\sigma = \alpha_{1}*n - \mu n \rho
\end{equation}

\begin{equation}
d = \gamma_{1}\rho
\end{equation}

Enzymes conducting decomposition make up a fraction of the total biomass, and the total biomass changes according to the growth rate. The total biomass $B$ defined in (\ref{eq:8}) multiplies by maximum rate of the Michaelis-Menion equation (\ref{eq:3}) to couple growth rate with decomposition rate. 

\begin{equation} \label {eq:8}
B = \int_{all x}\rho dx
\end{equation}

The rate of decomposition of carbon will then be dynamically effected by the growth of the fungus contrary to the typical assumption of a static $B$ as described by Schimel et al. (2003) \cite{Schimel2003}. Our formulation of $K$ which accounts for the proportion of biomass in the fungal growth containing the necessary enzyme for the carbon decomposition will change to account for this. 
\begin{equation} \label {eq}
K = r*S_{M}*S_{T}*0.04*G
\end{equation}
Where $r$ is a dimensionless parameter representing the relative abundance of one enzyme out of the total enzyme biomass of that specific fungus. The multiplication by $0.02$ is a generalization that for any given species of fungus, 2\% of the biomass will be relevant enzymes \cite{Moorhead2006}. $G$ is an empirically derived rate constant relating the mass of relevant enzymes to the maximum rate of the decomposition.

Limited carbon availability will also impact a fungi's growth rate. To account for this, we include a term multiplying $\nu$ that can be interpreted as the available carbon for the fungi to consume for growth. The first equation in the growth model then becomes
\begin{equation} \label{eq:10}
\frac{\partial \rho}{\partial t} = (1-LCI)*(10^{9})*C*n*\nu - d
\end{equation}
Where $LCI$ is the lignocellulose index\footnote{The lignocellulose index describes the ratio of non-hydrolysable lignin to hydrolysable holocellulose in litter} of a certain material and $(1-LCI)$ is then the proportion of the substrate carbon stored in holocellulose. Multiplying by $(1*10^{9}$ is approximation to convert grams of carbon to hyphae length in mm. [citation needed for g to mm] An investigation of the effects of including the carbon term in the growth model are addressed in a later section. \\

So our coupled growth and decomposition models can be summarized by equations (\ref{eq:2}), (\ref{eq:3}), (\ref{eq:8}), and (\ref{eq:10}).

\subsection{Defining Representative Enzymes}

Maynard et al. (2019) collected data for 8 relevant enzymes and utilized a standard clustering approach to create a subset of four enzymes which best represented the effects by the fungi's enzymatic makeup on competitive interactions \cite{Maynard2019}. Our model utilizes this open-source data by normalizing the production rates of each enzyme and calculating their relative abundance as a fraction of the total enzyme biomass for that specific fungus. Interactions caused by differing rates of decomposition among different enzymes in varying environmental conditions is then represented in our model by taking the weighted sum of all contributions to the decomposition rate from any one of four enzymes in a given fungi species, represented in the following equation
\begin{equation} \label{eq}
\frac{dC_{1}}{dt} = c_{1,a}\frac{de_{1,a}}{dt} + c_{1,b}\frac{de_{1,d}}{dt} + c_{1,c}\frac{de_{1,c}}{dt} + c_{1,d}\frac{de_{1,d}}{dt}
\end{equation}
Where $c_{1,a}$ is a enzyme breakdown efficiency coefficient for the fungal species $1$ and enzyme $a$. Here the rate $\frac{dC_{1}}{dt}$ is representative of the rate of decay of the substrate by the fungi species $1$. Breakdown efficiencies $c_{1,a}$ for each enzyme were derived from correlation coefficients drawn from experimentally observed data in Lustenhouwer et al. (2020) \cite{Lustenhouwer2020}. 

The total decomposition rate is then given by the sum of all rates over the various species of fungi.
\begin{equation} \label{eq}
\frac{dC_{tot}}{dt} = \sum_{i=1}^{n}\frac{dC_{i}}{dt}
\end{equation}
Where $n$ is the total number of fungi being simulated. Our model simulates a system in which all 35 fungi are growing and decomposing at the same time and interacting through competition for a limited nutrient supply.

\subsection{Estimating $\nu$ For Various Environments}\label{Environments Conditions}

When considering a sampling of decomposition rates in various environment types, we must determine how to estimate temperate and water potentials that ground fungi would experience. In the case of determining $\nu$, this requires finding projected temperature and water potential.

Amongst existing biome classification models, Whitaker's scheme \cite{Whittaker1970} is notably simple, providing a layout of biomes based on mean annual temperature and mean annual precipitation. However, modern classification of biomes has drifted away from using these traits as definitive biome identifiers \cite{Mucina2018}. These traits alone do not define all features of concern to fungal growth, such as soil composition for example. In addition, by Whitaker's scheme, we find that a given biome can have a wide range of average annual temperatures (such as arid desserts ranging from about -10 to 30 $^{\circ}C$) \cite{Whittaker1970}. We can sample a range of temperature and moisture values to output various $\nu$ and then select certain regions to profile in order to gauge potential environments where fungi can decompose and interact.
%We will sample within the temperature range of Whitaker's scheme: $-15^{\circ}$ to $30^{\circ}$ C of mean annual temperature.

The following are the specific environments selected to represent various biomes:

\end{multicols}

\begin{savenotes}
	\begin{table}[H]
		\begin{center}
			\begin{tabular}{|c c|} 
				\hline
				Biome & Specific Environment Selected \\ [0.5ex] 
				\hline\hline
				Dessert (Arid) & Sonoran Desert, USA \\ 
				\hline
				Grasslands/shrublands & (Semi-Arid), Shrubland, central Argentina\\
				\hline
				Temperate Forest & Sal Forests, Kumaun region, central Himalayas\\
				\hline
				Boreal Forest (Arboreal) & Pine Forests, Kumaun region, central Himalayas\\
				\hline
				Tropical Rain Forest & Tropical Forests, Barro Colorado Island, Panama \\
				\hline
			\end{tabular}
			\vspace*{-3ex}
			\captionof{table}{Specific Environments}\label{table1}
		\end{center}
	\end{table}
\end{savenotes}

Although water potentials can be approximated based on predictive models, the measure is best found experimentally from soil samples \cite{Abkenar2019}. Single water potential and annual temperature values were selected from ranges of values experimentally determined by a variety of field studies on these environments (see: Table 2). This decision was made since we're aiming to compare discrete environmental conditions rather than create a complete span of environmental conditions.

%% ACTUAL TABLE
\begin{table}[H]
	\begin{center}
		\begin{tabular}{|c c c|} 
			\hline
			Biome/ Environment & Temperature [$^{\circ}C$] & Water Potential [MPA]\\ [0.5ex] 
			\hline\hline
			Desert (Arid) & 15 & -4.5 \\ 
			\hline
			Grasslands/shrublands (Semi-Arid) &15.3 & -3.2\\
			\hline
			Temperate Forest & 12.49 & -1.09 \\
			\hline
			Boreal Forest (Arboreal)& 12.49 &  -1.51 \\
			\hline
			Tropical Rain Forest & 27.5 & -0.79 \\
			\hline
		\end{tabular}
		\vspace*{-3ex}
		\captionof{table}{Temperature \& Water Potentials}\label{table2}
	\end{center}
\end{table}

Note that these are not wholly representative configurations, but rather examples to provide insight into how fungal species with specific traits may respond in discrete and distinct conditions likely to exist. We can then find an approximate corresponding $\nu$ for each species based off each selected moisture potential and temperature, using experimental data from Maynard et al. 2019 where nonlinear least squares was used to fit a discrete data to a standard 3-parameter skew distribution \cite{Maynard2019}. These two $\nu$ values for each species and environment were then averaged to general one value for a given $\psi$ and $T$.

\begin{multicols}{2}	
\subsection{Other Responses to Temperature and Water Potential}\label{Other Responses to Temperature and Water Potential}

Our growth model takes into account temperature (T) and water potential ($\psi$) in two more parameters: the soil temperature coefficient ($S_T$) and the soil temperature coefficient ($S_M$). Moorhead et al. (1991) provides a simple relationship between $S_T$ and T using the rate of increase (Q):
\begin{equation}
\log_{10}(S_T) = \frac{T-25}{10}\log_{10}(Q)
\end{equation}
Although this equation does not take into account specific fungal response to temperature change, more recent evidence supports that this relationship is not direct, with most of the direct impact coming from moisture \cite{Petraglia2019}. As the ratio of rates of decomposition given a temperature change, Q as a parameter should represent the effective output of various mechanisms influenced by temperature rather than focusing on specific mechanisms. However, for the sake of simplicity, Q has been set standard constant to a value of 2.5 \cite{Moorhead1991}.

Water potential also relates to a constant, $S_M$, in a simple equation described in Moorhead et al. (1991) using $\alpha_2$ and $\lambda$:
\begin{equation}
S_M = \alpha_2 -\lambda \log_{10}(-\psi)
\end{equation}

\subsection{Assumptions}

Further assumptions made in the formulation of our model that have not yet been discussed are largely simplifications that are not relevant to any significant or interesting discussion. An itemized list of these assumptions is provided below.
\subsubsection{Growth}
\begin{itemize}
	\item Growth occurs in a 1-dimensional space
	\item Neglecting effects from direct interactions between fungi
	\item Neglecting effects from N limitation
	\item Litter is not added to or removed from the carbon pool by forces external of the fungi
	\item LCI is 0.291 and remains constant over time
	\item Dynamics regarding the rate of nutrient absorption and transport of metabolites within the
	mycelium are ignored or simplified *more specific
	\item Ignoring other growth traits
	\item Substrate use efficiency (SUE) is 1
	\item Hyphal death rate $\gamma_{1}$ remains constant regardless of environmental fluctuations
	\item 100\% of living hyphal tips are active
	\item No biomass created by a fungus will be geometrically isolated from the substrate
	\item Neglecting effects due to spacial limitations and phyiscal obstructions
\end{itemize}
\subsubsection{Decomposition Kinetics}
\begin{itemize}
	\item Neglecting effects due to enzymes not included in the model
	\item Production of enzymes increases proportionally with biomass
	\item For any species of fungus, 2\% of the biomass will be relevant enzymes
	\item Relative proportions of enzymes will stay constant regardless of environmental fluctuations
	\item Neglecting effects on decomposition rate due to changing metabolic rate of the fungi
\end{itemize}
\subsubsection{Temperature/Moisture effects}
\begin{itemize}
	\item Turgor is a function of water potential
	\item Fungal decomposition occurs in surface soil (0-30 cm)
	\item Experimental data for average water potentials in specific biomes/environments are representative of the conditions of those environments
	\item Temperature and moisture are the only environmental parameters of significance to the model
	\item Rate of increase of temperature $Q$ remains constant regardless of environmental fluctuations
	\item Soil moisture potential is effectively equivalent to that found in ground litter and wood fibers
\end{itemize}

\section{Parameter selection and representative result}
We created a representative run of the coupled growth and decomposition model to select realistic values for the parameterts, with some values found by consulting literature. An overview of the parameter selection for this representative run can be seen in Table \ref{table3}, however several parameters are worth some discussion.
 
\subsection{Growth model parameters}
Two main parameters were studied for their effect on the growth model's result: the branching rate and the hyphal death rate. These parameters were not agreed upon in literature so values were selected from a pool of possible valus depending on their effect on the model to converge on the most likely realistic output. 

Most results of the growth model were comprised of traveling wave solutions, converging to a uniform distribution in both space and time. This can be thought of as a convergence to the maximum growth of the fungus into its total space. The branching rate ($\alpha_{1}$) was found to increase the hyphal density in the end behavior of the solution, with the ending density increasing as $\alpha_{1}$ increases. The hyphal death rate ($\gamma_{1}$) was found to increase the oscillations in time and larger $\gamma_{1}$ values would increase the oscillations and the time to reach a given end-behavior. Many papers (Edelstein (1982), Lin et al. (2016), Schnepf et al. (2007), Du et al. (2019)) discussing the values of these parameters were concerned more with short-term dynamics in perfectly ideal conditions \cite{Edelstein1982, Lin2016, Schnepf2008, Du2019}. The purposes of this paper is the assess the longterm decomposition rates under variable conditions, so parameters that could predict long-term behavior, comparable to long term decomposition dynamics described in Moorhead et al. (2000), Moorhead et al. (2006), and Moorhead et al. (1991) were selected \cite{Moorhead1991, Moorhead2000, Moorhead2006}.

%\end{multicols}
%\begin{center}
%\includegraphics[width=\linewidth]{avg_contr_nu_environment.png}
%\captionof{figure}{Figure caption}
%\label{Fig 2.}
%\end{center}
%\begin{multicols}{2}

\subsection{Decomposition parameters}
The value for $G$ represents a rate constant relating the concentration of relevant enzymes to the maximum rate of carbon decomposition. Formulation for $S_{M}$ and $S_{T}$ come from Moorhead et al. (1991) which uses a simpler equation to describe the reaction dynamics for decomposition of carbon \cite{Moorhead1991}. Our model adds complexity to this so we based this parameter on agreement with other more similar models and experimental data. The most important conclusion drawn for $G$ was in selecting a timescale of decomposition to be comparable with that of Lustenhouwer et al. (2019) \cite{Lustenhouwer2020}.

%\end{multicols}
%\begin{center}
%\includegraphics[width=\linewidth]{biomass_growth_no_Cterm.png}
%\captionof{figure}{Figure caption}
%\label{Fig 2.}
%\end{center}
%\begin{multicols}{2}

\subsection{Representative results}
We ran the coupled growth and decomposition model using parameters summarized in Table \ref{table3} to obtain a representative result. In the growth model, $\rho$ was found to exhibit travelling wave dispersion throughout space, as similarly shown in Mimura et al. (2000) and Edelstein (1982) \cite{Mimura2000, Edelstein1982}. As a consequence of only running the simulation for a timespan of three years, the decrease in carbon over time looks fairly linear, with slight fluctuations likely due to the fluctuating fungal biomass densities over time. 

In the first initial runs, no coupling term defining a limited nutrient availability was present in the growth equations. This eliminated the effects of decreasing concentration of carbon as the growth model assumed an abundance of nutrients was present. Under short time scales this could be a perfectly valid assumption, as short time scales tended to show little overall decrease in the concentration of holocellulose carbon. However the consequences seen from introducing this term in the growth equations become more apparent as longer time scales are considered. These effects can be seen in our results in Figure 1. The outcome for overall biomass of the communitty was mostly unaffected other than replacing decaying oscillations towards a steady state with a somewhat constantly increasing biomass overtime with a decaying slope. 

Overall, including the nutrient availability term in the growth equation captures more of the fundamental interaction that this paper is concerned with (that of decomposing wood and ground litter rather than free-growing fungi), so it remained present for further analysis.

\end{multicols}
\begin{figure}[H]
\begin{subfigure}[t]{0.5\textwidth}
	\includegraphics[width=\textwidth]{biomass_growth_no_Cterm.png}
	\caption{Growth independent of carbon concentration.}
	\label{fig-a}
\end{subfigure}\hfill
\begin{subfigure}[t]{0.5\textwidth}
	\includegraphics[width=\textwidth]{biomass_growth_Cterm.png}
	\caption{Growth rate proportional to available carbon}
	\label{fig-b}
\end{subfigure}
\caption[Coupled/un-coupled fungi growth]{Fungal biomass of coupled growth and decomposition kinetics model} 
\label{Growth with and without coupling}
\end{figure}
\begin{multicols}{2}


\section{Analysis}

%This corroborates with various literature that finds soil moisture as a particularly influential environmental factor, influencing elongation through pressure gradients and testing of tolerance \cite{Maynard2019} \cite{Lustenhouwer2019}. %i think there may be a better citation, i am tired
%****In case linear regression added: The tropical rain forest environment, with a water potential of 0.76, lead to a }

\subsection{Total Carbon vs Time for Specific Environments} %direct_compare_decom_environmen.fig
A sampling of different environmental parameters within the model was taken as described in section \ref{Environments Conditions}. The results of these model runs were taken in terms of the total grams of carbon over time, shown in Figure 2. Here, we see more moist soil environments lead to clearly more expedient rates of total carbon decrease and reach complete carbon decomposition decomposition quicker in our model. The rain forest environment exceeds the other environmental profiles, effectively completing decomposition in under 1640 days (roughly 4.5 years. On this time scale of roughly 5 years individual biomes will exhibit negligible changes in environmental parameters. But biomes in a given area can change in non-negligible ways over longer time scales. We can consider these long term changes as the effective transition between different biomes and thus different overall performance, as assessed by the model. Therefore, difference in performance between different environments can be roughly equated to the change in performance under long term environmental fluctuations. 
% FAK: tervor succy boi

\end{multicols}
\begin{center}\label{Decay Environments}
\includegraphics[width=\linewidth]{direct_compare_decom_environment.png}
\vspace*{-6ex}
\captionof{figure}[Decomposition in five environments]{Total carbon in the substrate for five representative sampled environmental parameters.}
\end{center}
\begin{multicols}{2}

\subsection{$\nu$ vs Average Contribution to Substrate Decay} %avg_contr_nu_environment.fig

Using the decomposition model, average contribution to substrate decay and the tip elongation rate $\nu$ can be found for each fungal species for a given environmental configuration. 

The contribution of a particular fungi over time is taken to be the absolute value of the decrease in total carbon caused by all enzymes of the specific fungi in question divided by the total initial carbon of the substraight. As described in the model formulation, these decreases are calculated independently and summed together to calculate the total carbon decrease. The average contribution is then calculated as $C_{avg} = \frac{ \int C_{cont} dt}{\delta t}$ where $C_{cont}$ is the contribution of the enzyme in question and $\delta t$ is the total change in time.

\begin{ColumnFigure}\label{Competitive rankings}
	\centering
	\includegraphics[width=\linewidth]{avg_contr_nu_competative_rankings.png}
	\captionof{figure}[Contribution against $\nu$ with competitive ranking]{Average total decomposition contribution for specific fungi compared to hyphal extension rate. Coloring indicates competitive ranking coefficient.}
\end{ColumnFigure}

With one set of environmental parameters, we can gain insight into how different species of fungi potentially perform relative to each other, show in Figure 3. In addition to an expected loose positive correlation between $\nu$ and the average contribution of a given fungi, these is a lose correlation of higher fungi competitive ranking with greater $\nu$ and greater average contribution. This suggests that the fungi that are more 'active' in our model, with the largest performance dependence on $\nu$, are also the most competitive. Competitive ranking has a higher correlation with $\nu$, indicating that in our model growth rate may be more sensitive to $\nu$ in competitive circumstances. This aligns with established trends of the primary 2 fungal niches \cite{Maynard2019}.

\end{multicols}
\begin{center}\label{Environment Contributions}
\includegraphics[width=\linewidth]{avg_contr_nu_environment.png}
\captionof{figure}[Contribution against $\nu$ with five environments]{Average total decomposition contribution for specific fungi compared to hyphal extension rate. Coloring indicates values chosen for environmental parameters.}
\label{Fig 4.}
\end{center}
\begin{multicols}{2}

Results were also viewed under 5 different sets of environmental conditions (water potential and temperature), each corresponding to a different biome, shown in Figure 4. The results show that the correlation between $\nu$ and fungi contribution changes with environmental parameters. We find that in general, environments with a higher moisture potential lead to greater contributions overall and a greater contribution sensitivity to $\nu$ in our model. Our model agrees with the notion of water potential serving as an important limiting factor to the decomposition ability of individual fungi. These results can be interpreted as moisture potential and temperature limiting the effectiveness of faster-growing fungi.  

Species of fungi with a generally higher competitive advantage (corresponding to points with a higher $\nu$ and higher contribution) are much more adversely affected by changes in environmental conditions than points with a generally lower competitive advantage. These species with a higher competitive advantage would then have a smaller moisture niche width (environmental changes affect them more adversely). This relationship implies that species in this niche would have a lower moisture tolerance, compared with species of less competitive advantage and less sensitivity to $\psi$. %here is bit connected to moisture tolerance

\subsection{Short-Term Environmental Fluctuations}


Realistically, the external environment dictates environmental conditions of the system both in short time scales (year fluctuations for example) and long time scales. Due to the time scales of our model, the effects of very fast fluctuations (due to outside forces) on the scale of days are not considered as relative changes in the model's state have negligible changes on the order of a day. Thus we define our short term fluctuations to be roughly seasonal as this dictates the majority of local environmental variability. To study the effects of these short time scale fluctuations, we choose to implement an abstracted yearly moisture cycle, specifically that of the temperate forest environment. The cycle oscillates between the average yearly maximum and minimum values of $\psi$ twice a year, creating a rough yearly pattern of $\psi$ and thus $\nu$ for a given fungi from experimental data \cite{Zobel2001}\cite{Maynard2019}. Note that this moisture model is limited in assuming one period oscillation between a maximum and a minimum yearly, neglecting the reality of oscillating between local minimum in summer and winter and locate maximums in fall and spring. To incorporate this into the model, $\nu$ and $\psi$ were dynamically allocated over time based on the oscillatory behavior shown in Figure 5. 

Due to time limitations and scarce data on the parameters over these yearly oscillations, only two species of fungi were simulated. A comparison of their respective decomposition contributions with that of similar environmental conditions but static $\nu$ and $\psi$ is shown in Figure 6. There is no note-worthy new oscillatory motion seen in these results. There is also seen to be a slight increase in the effectiveness of both fungi in the oscillatory case. The interpretation of this increase can be that for time scales on the order of a year, the positives affect of increasing $\nu$ is greater than that of the negative affect of decreasing $\nu$ by the same amount. This would lead us to believe that faster oscillations of $\nu$ will generally have a positive net impact on the decomposition. The frequency limits of this principle would require further investigation.

\begin{ColumnFigure}\label{Oscillating nu contributions}
	\centering
	\includegraphics[width=\linewidth]{oscillating_psi_contribution.png}
	\captionof{figure}[Oscillating $\psi$ decomposition]{Contribution of two fungi species with yearly environmental fluctuations.}
\end{ColumnFigure}

\begin{ColumnFigure}\label{Oscillating nu}
	\centering
	\includegraphics[width=\linewidth]{oscillating_psi_nu.png}
	\captionof{figure}[$\psi$ \& $\nu$ cycle]{Environmental parameters for figure above over time.}
\end{ColumnFigure}

\subsection{Distribution of relative concentration of fungi}

The model allows us to adjust the relative concentrations of each fungi species being simulated. In the model, this is represented with a set of $0<c_{i}\leq 1$ such that $\sum c_{i} = 1$. Each one of these $c_{i}$ multiplies the rates of hyphal density increase and boundary/initial conditions in the growth model. With this metric, the relative concentrations of different fungi species are forced to satisfy the conditions $B_{i}/B_{tot}=c_{i}$ where $B_{i}$ is the biomass of the '$i$'th fungi species. This is an unrealistic constraint, as the relative concentrations of different fungal species should fluctuate with time as the different species of fungi grow at varying rates. But for the purpose of assessing the effect of biodiversity on the overall decomposition of the substraight, concentration fluctuation would add unneeded complexity to the analysis. \\
This analysis also considers the competitive rankings of the different fungi species \cite{Maynard2019}. The ranking-distribution coefficient, $a$ is then defined to be
\begin{equation}
a = \sum_{i=1}^{n} c_{i}*R_{i}
\end{equation}
Where $R_{i}$ s the competitive ranking for the '$i$'th fungi species. The ranking-distribution coefficient can be though of as a dot-product in $n$-dimensional space (where $n$ is the number of various fungi species being simulated). So $a$ will be highest when the $n$-dimensional vectors "line-up" in the sense that the distribution of fungi species is closest to the distribution of competitive rankings.\\
To study the effects of $a$ on the system overall, sample $c$ values had to be chosen. Due to a lack of literature results on simple, constant fungi distributions, arbitrary Gaussian distributions spread out over the different species of fungi were chosen. Because we are mainly concerned with the relation of any given distribution to the competitive rankings, the functions can be viewed as arbitrary or random distributions (because the ordering of fungal species in our Gaussian distribution has no physical meaning). The total carbon in the subtrate over time in the model runs using these various distributions is shown in Figure 7. 

\begin{ColumnFigure}\label{Carbon Decrease Fungi Distribution}
	\centering
	\includegraphics[width=\linewidth]{carbon_decrease_fungi_distribution.png}
	\captionof{figure}[Decomposition for different distributions]{Total carbon in substrate for sampled fungi species distributions.}
\end{ColumnFigure}

To assess the overall effectiveness of a single species distribution, the average carbon decrease over the whole interval was found via the equation $C_avg = \int (C_{0}-C)dt/(\delta t)$ where $C_{0}$ is our initial amount of carbon and $\delta t$ is the total time change over which the carbon decreases. The results of comparing this average carbon decomposed to the $a$ value for the species distribution in question is seen in Figure 8. Several interesting qualitative behaviors arise from the this comparison. Firstly, there is a clear smooth trend between $a$ and the average carbon decomposed up to approximately $a=0.65$. At this point the smooth trend completely vanishes and the data points appear to be randomly distributed in the area. We hypothesize that this critical value of $a$ could be representative of a bifurcation point; however further analysis on the dynamics of these coupled systems interacting would need to be done to investigate this hypothesis further. Additionally we notice that at this critical point, the average carbon decomposed tends to grow significantly. The physical interpretation of this critical point is deeply related to the effectiveness of the competitive rankings as they relate to a biologically diverse system. The results show that beyond a certain point, distributing the biodiversity more closely to the competitive rankings does not have any noticeable effect on the overall effectiveness of the system. \\
This leads to two possibilities beyond the critical point for the system. One possibility is that optimum biodiversity is based on more complex dynamics than the competitive rankings can assess. The other possibility is that the system behaves in a chaotic or random nature beyond this critical point and no clear metric of biodiversity could be correlated to the effectiveness of a given distribution.

\end{multicols}
\begin{center}\label{Competitive Distribution Coefficients}
\includegraphics[width=\linewidth]{a-competative-distribution-coef_to_avg_Clost.png}
\captionof{figure}[Average total decomposition against a]{Average total decomposition as seen in Figure 8 plotted against ranking distribution coefficient for a certain species distribution.}
\end{center}
\begin{multicols}{2}

\section{Limitations}
\vspace*{-2ex}
\subsection{Growth Coupling}
The coupling between our growth model and our decomposition kinetics model was scarsely found in literature. Instead, literature would in some cases adopt a simpler mechanism to describe fungi growth to match decomposition kinetics of equal complexity to our model, or would in other cases implement a more complex mechanism for fungi nutrient uptake while neglecting much of the complexity from interactions between fungi and with the larger environment. This resulted in a lack of applicable parameters from literature to choose from for our coupled model, for which we decided either to pull parameter values from simpler models described above or to empirically derive our parameters by fitting the model output to other reasonable outputs (as was done with the parameter $G$). 

\subsection{Decomposition kinetics}
In our comparison of different methods for coupling the growth equation to the decomposition kinetics, there are some unrealistic disagreements between the early behavior of each growth model; mainly that of a slight increase in the first maximum biomass reached by the carbon limiting growth model. Expected behavior would exhibit less biomass gained over time when the nutrient level is limited. This result is likely a consequence of bad parameter choices derived with imperfect methods described above or a failure to incorporate a more realistic relationship between the decomposed carbon pool and available carbon pool from which the fungi supplement their own nutrient supply.

The Michaelis-Menten dynamics we implemented often considers enzyme concentration to be proportional to the maximum rate of the reaction. Taking this concentration to be proportional to total fungal biomass in some set region of space does not account for the concentration of enzymes becoming saturated as the fungi spread farther out through space. A more realistic incorporation of this idea into the model would need to address the geometric complexities involved in examining only one region of a much larger fungi growth. Even further unpackaging of these dynamics would account for fungal growth into areas that do not have physical access to the carbon being decomposed. At this point, a simplified one-dimensional growth model would not suffice.

The effect of considering biomass proportional to concentration would be seen in the longer term behavior of the model, as the fungi growth becomes limited. We may expect to see more complex dynamics transitioning between the "unbounded" fungi growth and the steady-state behavior of fungi growth. Thus, the largest area of error for our results resulting from this assumption is likely in the area of time transitioning unbounded growth and steady-state behavior.

Values for our parameter, $K_e$ were difficult to find in the literature, as the dynamics of the four enzymes considered were not agreed upon in literature. Additionally, cellobiohydrolase was sometimes found to follow different dynamics than the Michaelis-Menten equation \cite{Razavi2015}. To obtain a more accurate value for this parameter, a study of the model's overall sensitivity to varying $K_e$ should be conducted. A more accurate representation of fungi interactions would include a larger subset of enzymes and account for more distinctive differences in decomposition dynamics between various enzymes. But any study of alternative decomposition kinetics between enzymes was not done due to time limitations. 

\subsection{Direct Interactions}

It was discussed in literature that there are competitive direct interactions observed between fungi growing in the same environment by various mechanisms \cite{Boddy2000}. Our model ignored the effects of these direct interactions and instead focused on indirect interactions via competition for the same nutrients source. While some effects from offensive skirmishes between fungi are typically taken to be negligible by literature other than an increase in metabolic rate, the effects from competitve fungi obstructing the growth to others due to space-limitations would likely be non-negligible. Essentially, our simulation only considered fungi competing for food, when in reality, they compete for food and space. 

Further review would need to be done to assess the space-limitations inherited by environments other than open ground litter which does not pose obvious obstructions to fungi growth. This assumption of negligible spacial-limitations would not apply in the case of geometric isolation of particular fungi. Moreover, geometric access by fungi to the substraight is another complexity that would need to be addressed in a higher-dimensional growth model. 

\subsection{Environmental Fluctuations}

The environmental conditions were analyzed for short term and long term fluctuation (short term being yearly cycles and long term being overall biome transitions over time). Our analysis found that oscillating environmental conditions showed slightly improved decomposition rates for two species of fungi, but additional studies involving a more diverse subset of species of fungi that might reveal niche complementarity were not studied due to time limitations.

In this analysis, only the effects from varying $\psi$ were studied. Realistic environmental fluctuations would see both $T$ and $\psi$ fluctuating simultaneously, but for the purpose of studying the general effect of oscillating environmental conditions, this can be taken as a representative result. Future studies of the model's response to fluctuating environmental conditions could include a study of the system's response to a continuum of frequencies for environmental fluctuations, as well as studying the effects of fluctuating multiple environmental conditions simultaneously. 

The correlations found between distributions of fungi species and overall performance of the system in the static environmental conditions of the Arid climate biome is poorly understood and would require further investigation or numerical fitting to be studied in more detail. The change occurring at the critical point may also be a result of numerical error in approximate solution to the differential equations being solved (although there is no other indication that this is the case). Further analysis of this correlation could include assessment of this correlation in different biomes and in oscillating environmental conditions. 

\section{Conclusion}
Advantages of certain species were taken into account via their relative concentrations of different enzymes as well as in their dependence of $\nu$ on temperature and moisture. These advantages were found to have a positive correlation with the contribution of different species to the overall decomposition. However, this exact relationship is unclear from the data. The competitive ranking of different fungi species was found to correlate to a higher sensitivity on $\nu$ in relation to the species' contribution to decomposition. Distributions according to these rankings in our model were found to positively impact overall system performance up to a critical point, where the trend dissolves completely. The presence of short term environmental fluctuations were found to have a slight positive impact on performance for a small sample of fungi species simulated. Long term environmental fluctuations, in the form of transforming overall biomes, indicates that overall fungi decomposition productivity should generally decrease as more arid conditions with less available moisture are approached.

The dynamics of different biodiversity distributions is complex, resulting in limited scope of certain descriptive values, such as competitive ranking, in capturing the full efficiency of the system. For a given species, an increase in hyphal extensional rate corresponds to different relative increases in efficiency based on environmental conditions as well as the species' sensitivity to those conditions. Generally, increasing $\nu$ has a greater affect for species with a lower moisture tolerance in an optimal environment and a lesser affect for species with a lower moisture tolerance in a sub-optimal environment. Additionally, the results would imply that species with a lower moisture tolerance would perform significantly worse under long term fluctuations compared with species of a higher moisture tolerance, which would perform only slightly worse. The effect of short term fluctuations on species with a lower moisture tolerance would require further study to discern. 

%\begin{ColumnFigure}
%	\centering
%	\includegraphics[width=\linewidth]{avg_contr_nu_competative_rankings.png}
%	\captionof{figure}{my caption of the figure}
%\end{ColumnFigure}
\end{multicols}
\newpage
\appendix
\section{Model Parameters}

Here are the model parameters for the coupled growth and decomposition model for Armillaria gallica located at 30.465247 degrees latitude and -89.040298 degree longitude secreting cellobiohydrolase (Cel7A) to decompose hardwood holocellulose \cite{Maynard2019, Kari2014}:

\begin{savenotes}
	\begin{table}[ht]
		\begin{center}
			\begin{tabular}{|c c c c c|} 
				\hline
				Parameter & Symbol & Value & Units & Source and Specification \\
				\hline\hline
				Half-Saturation constant \footnote{Also called Michaelis Constant.} & $K_e$ & 7 & $\frac{g_{enzyme}}{L_{litter}} $ & \cite{Kari2014} Enzyme \\ 
				\hline
				Holocellulose carbon \footnote{We assume that all carbon compounds excluding lignin are holocellulose.} & $1-LCI \footnote{Where LCI is the lignocellulose index.}$  & 0.709 & N/A & \cite{Segato2014} \\ %confirm unitless
				\hline
				Hyphal tip elongation rate& $\nu$& 0.250 & $\frac{mm}{day}$ & \cite{Maynard2019} Species, $\psi$, T\\
				\hline
				Temperature & T & 25 & $^{\circ}C$ &\cite{Maynard2019} Specie's habitat\\
				\hline
				Water potential & $\psi$ & -0.5 & MPa &\cite{Maynard2019}\\
				\hline
				Enzyme biomass ratio \footnote{Proportion of specific enzyme biomass to total enzyme biomass.} & r & 0.437 & $\frac{g}{g}$ &\cite{Maynard2019} Species\\
				\hline
				Hyphal death rate& $\gamma_1$ & 0.15 & $day^{-1}$ &\cite{Schnepf2008}\\
				\hline
				Anastomosis coefficent & $\mu$ & 0.3 & $\frac{mm}{day}$ &\cite{Lin2016}\\ % Trevor sussed on units
				\hline
				Branching rate & $\alpha_1$ & 1.2372 & $day^{-1}$ &\cite{Du2019}\\
				\hline
				Intercept of $S_M$ function \footnote{The intercept of soil moisture effect on decay rate.}& $\alpha_2$ & 0.311 & N/A &\cite{Moorhead1991}\\
				\hline
				Slope of $S_M$ function & $\lambda$ & 0.345 & $N/A$ &\cite{Moorhead1991}\\ % units? might be MPa^{-1}, but comes out of log?
				\hline
				Soil moisture coefficient & $S_M$ & 0.4149 & N/A &\cite{Moorhead1991} $\psi$\\ % mention equation later in paper
				\hline
				Soil temperature coefficient & $S_T$ & 1 & N/A &\cite{Moorhead1991} T\\
				\hline
				Rate of Increase & $Q$ & 2.5 & $^{\circ}C$ &\cite{Moorhead1991} T\\
				\hline
				Rate constant \footnote{Proportionality constant between maximum rate of decomposition and enzyme biomass.} & G & 10 & $g*mm^{-1}*day{-1}$ &\cite{Lustenhouwer2020}\footnote{Emprically derived.}\\  %value TBD, empirically derived, Trevor will justify
				\hline
			\end{tabular}
		\vspace*{-3ex}
		\captionof{table}{Model Parameters}\label{table3}
		\end{center}
	\end{table}
\end{savenotes}


\section{Appendix Tables}

\begin{table}[H]
	\begin{center}
		\begin{tabular}{|c c c|} 
			\hline
			Biome/ Environment & Average Annual Temperature [$^{\circ}C$] & Selected Value [$^{\circ}C$]\\ [0.5ex] 
			\hline\hline
			Desert (Arid) & 10 to 20 \cite{Davey2007} & 15 \\ 
			\hline
			Grasslands/shrublands (Semi-Arid) & 15.3 \cite{Pelaez1994} & 15.3 \\
			\hline
			Temperate Forest & -1.42 to 26.39 \cite{Zaz2019} & 12.49 \\
			\hline
			Boreal Forest (Arboreal)& -1.42 to 26.39 \cite{Zaz2019} & 12.49 \\
			\hline
			Tropical Rain Forest & 23 to 32 \cite{Paton2019} &27.5 \\
			\hline
		\end{tabular}
	\vspace*{-3ex}
	\captionof{table}{Environment Temperature Data}\label{table6}
	\end{center}
\end{table}



%NEED TO EDIT TABLE WITH FOOTNOTES TO FIT (put spec envs?)
%For himalayas, consider putting all values as footnotes
\begin{savenotes}
	\begin{table}[H]
		\begin{center}
			\begin{adjustbox}{width=1\textwidth}
			\begin{tabular}{|c c c|} 
				\hline
				Biome/ Environment & Water Potential Range [MPa] & Selected Value [MPa] \\ [0.5ex] 
				\hline\hline
				Desert (Arid) & -4.0 to -5.0 MPa \cite{Nilsen1983} & -4.5 MPa \\ 
				\hline
				Grasslands/shrublands (Semi-Arid) & -1.4 to -5.0 \footnote{Measurements taken November through January at 100 cm soil depth.} \cite{Pelaez1994} & -3.2 MPa\\
				\hline
				Temperate Forest & -0.44 (Fall), -1.19 (Winter), -0.58 (Spring), & -1.09 MPa\\
				&-1.42 (Early Summer), -1.81 (Summer) \cite{Zobel2001}&\\
				\hline
				Boreal Forest (Arboreal) & -0.83 (Fall), -1.20 (Winter), -0.55 (Spring),& -1.51 MPa\\
				&-1.61 (Early Summer), -3.36 (Summer) \cite{Zobel2001} &\\
				\hline
				Tropical Rain Forest & -1.57 MPa to 0.00 MPa \cite{Kupers2019} & -0.79 \\
				\hline
			\end{tabular}
			\end{adjustbox}
		\vspace*{-3ex}
		\captionof{table}{Environment Water Potential Data}\label{table4}
		\end{center}
	\end{table}
\end{savenotes}

\begin{savenotes}
	\begin{table}[H]
		\begin{center}
			\begin{adjustbox}{width=1\textwidth}
			\begin{tabular}{|c c c c c|} 
				\hline
				Parameter & Symbol & Value & Units & Source and Specification \\
				\hline\hline
				Half-Saturation constants \footnote{Also called Michaelis Constant.} & $K_e$ & - & mM &  Enzyme \\ 
				& $K_e$ - Phenol oxidase & 0.89 & mM &  \cite{Davidson2012}\\
				& $K_e$ - Phosphatase & 0.94 & mM &  \cite{Nannipieri2011}\\
				& $K_e$ - Peroxidase & 0.7475 & mM &  \cite{Chance1943}\\
				& $K_e$ - Cellobiohydrolase & 13.90 & mM &  \cite{Razavi2015}\\
				\hline
				Hyphal tip elongation rate& $\nu(T,\psi)$& Various & $\frac{mm}{day}$ & \cite{Maynard2019} Species, $\psi$, T\\
				\hline
				Soil moisture coefficient & $S_M$ & $S_M = \alpha_2 -\lambda \log_{10}(-\psi)$ & N/A &\cite{Moorhead1991} $\psi$\\ % mention equation later in paper
				\hline
				Soil temperature coefficient & $S_T$ & $\log_{10}(S_T) = \frac{T-25}{10}\log_{10}(Q)$ & N/A &\cite{Moorhead1991} T\\ 
				\hline
			\end{tabular}
			\end{adjustbox}
		\vspace*{-3ex}
		\captionof{table}{Environmental \& Parameter Table}\label{table5}
		\end{center}
	\end{table}
\end{savenotes}

\vspace*{-4ex}
\section{Appendix Figures}

\begin{center}\label{Rainbow Spaghetti}
\includegraphics[width=0.6\linewidth]{contributions_over_time(rainbowspaghetti).png}
\captionof{figure}[Contribution of 35 fungi]{Decomposition contribution of 35 simulated fungi in static arid environmental conditions.}
\end{center}

\begin{center}\label{Rho Growth 3D}
\includegraphics[width=\linewidth]{rho_growth_no_Cterm.png}
\captionof{figure}[Uncoupled growth model output]{Hyphal density in time and space for non carbon-limited growth model.}
\end{center}

%\end{multicols}
\newpage
Natural ecological systems rely on an expanse of cycling, from carbon to energy. Materials pass in and out of organisms and throughout environments, each piece occupying niches that play roles in determining the total structure.

Fungi, a group of eukaryotic organisms \footnote{"Eukaryotes are unicellular or multicellular organisms, which have membrane enclosed organelles such as specially nucleus, mitochondria, golgi apparatus and chloroplasts in plants". \cite{Panawala2017}}, are one such piece that hold an integral role in this continuous system. They serve as the interface between other living organic material and the soil with a important job: decomposition.

Decomposition is the process of breaking down larger substances, called substrates, into smaller molecules that can be realised into the soil or air. Without decomposers like fungi, the nutrient cycle would be halted. This role has been long established \cite{Kakde2009}. However, more precise mechanisms and models for fungal decomposition are being progressively developed. In particular, we have built a greater understanding of more specialized fungal niches and response to environmental changes on various time scales.

Within the role of decomposer, there are yet more divisions of work. Sepcific species of fungi are abel to sucrete different enzymes capable of breaking down distinct substrates. Alone, a single species of fungi would not be able to break down a log, but a community of fungi will. Amongst a variety of specific fungal roles, there emerges a pattern of two factions of fungal types. %need better word than factions

The first type if that of the competitive, fast growing fungi. In the right conditions, these fungi grow and decompose material faster than the other grouping. This fast growth is indicated by the hyphal extension rate. Fungi are composed of long interwoven strands of filament, know as hyphae, that grow by extending outwards from Spitzenkörper - a vesicle bundle located at the tip of each strand \cite{Edelstein1982}. Given this growth nature, hyphal extension rate can be used as a proxy measure for overall fungal growth. These species are opportunists, meaning they seek out specific conditions in which to seize the opportunity to grow. Indeed, these species are particularly environmentally dependent. Despite generally having these higher hyphal extension rates, in order to reach there competitive and productive potential they require more optimal growth conditions. For example, if the species are place in an environment with limited access to water, meaning there is low moisture potential, often there advantage over the other species is little to none.

By comparison, the advantage of the second group of fungi relies more on their resilience in a variety of environmental conditions rather than in there speed of growth. When placed in environments with more or less optimal moisture and temperature, they maintain a roughly consistent growth rate, increasing growth to more favorable conditions at a fraction of the increase demonstrated by fast-growing fungi.

Both niches are significant to a fluctuating environment. Having a biodiversity of fungal types supports the entire niches ability to break down organics matter, with opportunists making for quicker decomposition as possible and slower-growing fungi serving as a steady, more-permanent  source.

In addition to our understanding of this further niche differentiation, we have also gained insight into how these environmental fluctuations themselves may impact decomposition. Changes on short time scales, such as seasonal variations of temperature and moisture availability, have potentially impacted the system by both creating incentive for a diverse array of fungal species. When this variation was implemented, it resulted in increased growth rates.

On larger time scales, climate and geological changes care result in evolving biomes. This can be thought of as a common general environmental configuration transforming into another, such a a temperate forest transitioning into and arid dessert or a tundra transitioning into a grassland as global temperatures increase. In general, of environments examined, fungal species tended to grow quicker in environments with higher water potentials. For example, a sampled tropical rain forest contained significantly faster growth rates than other sampled biomes. This agrees with the rain forests' relatively increased rate of nutrient cycling, with high average plant growth \cite{Lodge1993}.

Interactions between members of ecological systems are complex and intricate: there are many mechanisms of fungal decomposition yet to be understood and investigated. However, our understanding of fungi and their roles will continued to be studies, as more advanced techniques for both modeling interactions both experimentally and mathematically.

\begin{center}\label{Competitive Distribution Coefficients in article}
	\includegraphics[width=\linewidth]{avg_contr_nu_competative_rankings.png}
	\captionof{figure}[Decomposition contribution against $\nu$ with competitive ranking]{Hyphal extension rate, $\nu$, compared with resulting average contribution to decomposition for 35 different species of fungi studied in Maynard et al. (2019) in 5 different sample biomes with varying temperatures and soil moisture potentials, demonstrating the importance of environmental conditions to more competitive species.}
\end{center}

\newpage
\printbibliography
\end{document}
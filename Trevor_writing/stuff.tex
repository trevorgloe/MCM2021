\documentclass[10pt]{article}

\usepackage[utf8]{inputenc}
\usepackage[T1]{fontenc}


\usepackage{amssymb}
\usepackage{mathpazo}
\usepackage{enumitem}
\usepackage{amsthm}

\title{MCM2021}


\usepackage[margin=2cm]{geometry}

\begin{document}
\maketitle

\section*{Growth model}
Fungal growth models typically focus on the increase in overall biomass due to hyphal extension. As described by [Edelstein 1982], the growth of hyphae over time depends on the spacial density of hypae tips (from which hyphae can extend and grow), the tip extension rate and the hyphal death rate. The tips are created by various forms of branching. This model is summed up by the following equations:

\begin{equation} \label {eq}
    \frac{\partial \rho}{\partial t} = n\nu - d
\end{equation}
\begin{equation} \label {eq}
    \frac{\partial n}{\partial t} = -\frac{\partial n\nu}{\partial x} + \sigma
\end{equation}
These equations form a basis for many other models, typically expanding on the effects of nutrient concentrations in the decomposing substrate. The largest assumption made by this model that we will differ from is the assumption that the mycellium grows over an unlimited nutrient supply. Both [Edelstein 1983 Growth and Metabolism in Mycelial Fungi] and [Davidson and Park, 1997] expanded on this model to include interactions with an external substrate by considering concentrations of nutrients both inside and outside the fungi. 
\\ Another note-worthy growth model is described in [Du, Tran, Perre 2019]. This more recent model describes the growth of the fungus as a 3-dimensional reaction-diffusion equation based on microscopic growth mechanisms, also in terms of the tip density, tip extension rate, hyphal branching, and hyphal elimination rates. However this model also has an added term, the proportion of active tips. In all models, the terms for $\nu$, $\sigma$ are taken to be constants given from experimental data. As noted in [Edelstein 1982], these parameters are seen to depend on nutrient availability and environmental factors in practice. In the review of various fungal growth models described in [Bayesian model selection framework for identifying growth patterns in filamentous fungi, 2016], a specific method of modeling branching and hyphal death was found to match experimental observations in the widest variety of settings. Based on these results we will choose to implement the following relationships, comprising dichotomous branching, tip-hypha anastomosis, and hyphal death (or YHD). 
\begin{equation} \label {eq}
    \sigma = \alpha_{1}*n - \mu n \rho
\end{equation}

\begin{equation} \label {eq}
    d = \gamma_{1}\rho
\end{equation}
To 
Based on various studies (ADD CITATIONS), we have found a clear dependence of the tip elongation rate, $\nu$ on environmental conditions. For the purpose of this model, we will  define a relationship between 


\section*{Decomposition kinetics}
As is described in [Moore and Classen, 2015] [Edelstein 1983 Growth and Metabolism in Mycelial Fungi] [Moorhead and Sinsabaugh, 2006] [Parnas 1974] [Schimel and Weintraub 2003], the decomposition of a certain substance under the influence of a microbial decomposer is a function of the current concentration of the said substance by the Michaelis–Menton equation. In [Schimel and Weintraub 2003] and [Moorhead and Sinsabaugh, 2006], the maximum rate of this decomposition is proportional to the concentration of enzymes acting on it. Here, we make the assumption that the enzyme concentration is proportional to the concentration of biomass (fungi in our case) in our system. We will thus describe the decompostion of carbon in the substrate by the following equation.
\begin{equation} \label{eq}
    \frac{dC}{dt} = \frac{KBC}{K_{e}+C}
\end{equation}
The constant $K=rS_{M}S_{T}$ is taken to be the proportion of a specific enzyme to the total biomass of a specific species of fungi multiplied by two environmental limiting factors, as recommend in [Schimel and Weintraub 2003]. These limited factors are used by [Moorhead and Reynold 1990] to multiply the rates of carbon uptake from microorganism, although the model used a slightly simpler first order, linear equation. Here we use it in a similar way in limiting our maximum decay rate. The factors, $S_{M}$ and $S_{T}$ are computed by the following equations
\begin{equation} \label{eq}
    S_{M}=\alpha_{2} - \lambda \log_{10}(-\psi)
\end{equation}
\begin{equation} \label {eq}
    S_{T} = XXXXX
\end{equation}


\section*{Parameters}
\begin{center}
\begin{tabular}{ c c c c }
\nu & \alpha_{1} & \mu & \gamma_{1} \\
0.28 cm/d & 0.74 d^{-1} & 0.15 cm^3m^{-1}d^{-1} & 0.73 d^{-1}\\
\end{tabular}
\end{center}

\section*{Coupled growth and decomposition model}
To see the dynamics effects of the fungal growth on the rates of decomposition reactions (via enzymes in the fungi) the above models are coupled such that the total biomass found from the growth model will multiply the maximum rate of the Michaelis-Menten reaction equation. The  relationship can be summarized as follows
\begin{equation} \label {eq}
    B = \int_{all x}\rho dx
\end{equation}
The rate of decrease of carbon will then be dynamically effected by the growth of the fungus, as apposed the the typical assumption of a static $B$, such as that described by [Schimel 2003]. The constant $K$ will then have to be accounting for the proportion of biomass in the fungal growth containing the necessary enzyme for the carbon decomposition, as the Michaelis-Menten equation is based on the concentration of the catalyzing enzyme. Thus our formulation of $K$ will change to account for this. 
\begin{equation} \label {eq}
    K = r*S_{M}*S_{T}*0.04*G
\end{equation}
Where $r$ is the dimensionless parameter relating the total biomass of enzymes in a species of fungus to the biomass of that specific fungus. The multiplication by $0.02$ is a generalization that for any given species of fungus, 2 percent of the biomass will be relevant enzymes [Moorhead 2006]. The rate constant $G$ relates the mass of relevant enzymes to the maximum rate of the decomposition reaction and is empirically derived (see later section). \\
A more realistic coupling of the rates of carbon decomposition and the fungal growth should include the effect of available carbon on the fungi's growth rate. To account for this, we include a term multiplying $\nu$ that can be interpreted as the available carbon for the fungi to consume for growth. The first equation in the growth model then becomes
\begin{equation} \label{eq}
    \frac{\partial \rho}{\partial t} = (1-LCI)*(10^{-9})*C*n*\nu - d
\end{equation}
Where $LCI$ is the lignocellulose index of the certain material and thus $(1-LCI)$ is approximately the proportion of the substrate carbon stored in holocellulose. The multiplication by $(1*10^{-9}$ is approximation to convert grams of carbon to mm for hyphae. An investigation of the effects of including the carbon term in the growth model are addressed in a later section. \\
So our coupled growth and decomposition models can be summarized by equations (10), (2), (5), and (8).

\section*{Parameter selection and representative result}
To select realistic values for the parameters, we created a representative run of the coupled growth and decomposition model, consulting the literature to find parameter values. An overview of the parameter selection for this representative run can be seen in (parameter table), however several parameters are worth some discussion. 
\subsection*{Growth model parameters}
The effect of two main parameters were studied for their effect on the growth model's result: the branching rate and the hyphal death rate. These parameters were not agreed upon in the literature so their effect on the model output was evaluated and they were selected from a pool of possible values based on the most likely realistic output. Most results of the growth model were comprised of traveling wave solutions, converging to a uniform distribution in both space and time. This can be thought of a the maximum growth of the fungus into it's total space. The branching rate ($\alpha_{1}$) was found to increase the hyphal density in the end behavior of the solution, with the ending density increasing as $\alpha_{1}$ increases. The hyphal death rate ($\gamma_{1}$) was found to increase the oscillations in time and thus larger $\gamma_{1}$ values would increase the oscillations and the time to reach a given end-behavior. Many of the papers discussing the values of these parameters [Edelstein 1982] [Lin 2016] [Schnepf 2007] [Du 2019] are concerned more with short term dynamics in perfectly ideal conditions. The purposes of this paper is the assess the long term decomposition rates under different conditions, so parameters that can predict long-term behavior, comparable to long term decomposition dynamics described in [Moorhead 2000] [Moorhead 2006] [Moorhead 1991] were selected.
\subsection*{Decomposition parameters}
The value for $G$ is interpreted as a rate constant, relating a concentration of relative enzymes to the maximum rate of carbon decomposition. formulation for $S_{M}$ and $S_{T}$ come from [Moorhead 1991] and in said model, a simpler equation is used for the reaction dynamics for decomposition of carbon. Because of this, we cannot use the same rate constant and must base this parameter of agreement with other models or experimental data. The most important aspect we examined for the section of $G$ was the timescale of decomposition to be comparable with that of [Lustenhouwer 2019].
\subsection*{Representative results and model comparison}
As a representative result of the coupled growth and decomposition model, the two were run using the parameters summarized in (parameter table). In the growth model, $\rho$ was found to exibit travelling wave dispersion throughout space, as similarly shown in [Mimura 2000] and [Edelstein 1982]. Due to the simulation only running for a timespan of 3 years, the decrease in carbon over time looks fairly linear, with slight fluctuations likely due to the fluctuating fungal biomass densities over time. The first initial runs were done without the coupling term in the growth equations, so the growth model was assuming an abundance of nutrients and not affected by the decreasing concentration of carbon. Additional runs were completed in similar parameter conditions, but including the coupling term in the growth equation. The effects of including this term can be seen in (figures biomass-growth-Cterm and biomass-growth-no-Cterm). Physically, the assumption that there is an abundance of nutrience available for fungal growth and that the nutrients does not change over time is more accurate near the beginning of the time span. However the differences seen in including the nutrient availability term in the growth equations become more apparent as longer time scales are considered. Our comparative model results reflect this as early behavior of the two biomasses is identical. Overtime the difference between the two models becomes more prevalent and the behavior of the growth model including the coupling term becomes distinct. The difference in qualitative behavior in the relevant time scale of our model gives validation to the necessity of including the coupling term, as it shows that the assumption of nutrient surplus becomes unrealistic. Additionally, including the nutrient availability term in the growth equation captures more of the fundamental interaction that this paper is concerned with (that of decomposing wood and ground litter rather than free-growing fungi), so it was included in further analysis. There is some unrealistic quantitative disagreement between the two models in their early behavior likely due to parameter adjustments needed  due to adding the coupling term.


\section*{Fungal species interaction}
Due to the complex mechanisms of direct interaction between fungi, we chose to focus not on the direct interactions between the fungi in their consumption of the substraight, but on the interactions caused by differing rates of decomposition among different enzymes in varying environmental conditions and how the relative amounts of different bacteria in a fungi species will impact its competition. This was formulated in the model by taking the weighted sum of all contributions to the decomposition rate from any one of four enzymes in a given fungi species, represented in the following equation
\begin{equation} \label{eq}
    \frac{dC_{1}}{dt} = c_{1,a}\frac{de_{1,a}}{dt} + c_{1,b}\frac{de_{1,d}}{dt} + c_{1,c}\frac{de_{1,c}}{dt} + c_{1,d}\frac{de_{1,d}}{dt}
\end{equation}
Where $c_{1,a}$ is a enzyme breakdown efficiency coefficient for the fungal species $1$ and enzyme $a$. Here the rate $\frac{dC_{1}}{dt}$ is representative of the rate of decay of the substrate by the fungi species $1$. The total decomposition rate is then given by the sum of all rates over the various species of fungi.
\begin{equation} \label{eq}
    \frac{dC_{tot}}{dt} = \sum_{i=1}^{n}\frac{dC_{i}}{dt}
\end{equation}
Where $n$ is the total number of fungi being simulated. 

\section*{Analysis}
\subsection*{Distribution of relative concentration of fungi}
The model allows us to adjust the relative concentrations of each fungi species being simulated. In the model, this is represented with a set of $0<c_{i}\leq 1$ such that $\sum c_{i} = 1$. Each one of these $c_{i}$ multiplies the rates of hyphal density increase and boundary/initial conditions in the growth model. With this metric, the relative concentrations of different fungi species are forced to satisfy the conditions $B_{i}/B_{tot}=c_{i}$ where $B_{i}$ is the biomass of the '$i$'th fungi species. This is an unrealistic constraint, as the relative concentrations of different fungal species should fluctuate with time as the different species of fungi grow at varying rates. But for the purpose of assessing the effect of biodiversity on the overall decomposition of the substraight, concentration fluctuation would add unneeded complexity to the analysis. \\
This analysis also considers the competitive rankings of the different fungi species \cite{Maynard2019}. The ranking-distribution coefficient, $a$ is then defined to be
\begin{equation} \label{eq}
    a = \sum_{i=1}^{n} c_{i}*R_{i}
\end{equation}
Where $R_{i}$ s the competitive ranking for the '$i$'th fungi species. The ranking-distribution coefficient can be though of as a dot-product in $n$-dimensional space (where $n$ is the number of various fungi species being simulated). So $a$ will be highest when the $n$-dimensional vectors "line-up" in the sense that the distribution of fungi species is closest to the distribution of competitive rankings.\\
To study the effects of $a$ on the system overall, sample $c$ values had to be chosen. Due to a lack of literature results on simple, constant fungi distributions, arbitrary Gaussian distributions spread out over the different species of fungi were chosen. Because we are mainly concerned with the relation of any given distribution to the competitive rankings, the functions can be viewed as arbitrary or random distributions (because the ordering of fungal species in our Gaussian distribution has no physical meaning). The results of model runs using these various distributions is shown in (figure carbon-decrease-fungi-distribution). To assess the overall effectiveness of a single species distribution, the average carbon decrease over the whole interval was found via the equation $C_avg = \int (C_{0}-C)dt/(\delta t)$ where $C_{0}$ is our initial amount of carbon and $\delta t$ is the total time change over which the carbon decreases. The results of comparing this average carbon decomposed to the $a$ value for the species distribution in question is seen in (figure a-competative-distribution-coef-to-avg-Clost). Several interesting qualitative behaviors arise from the this comparison. Firstly, there is a clear smooth trend between $a$ and the average carbon decomposed up to approximately $a=0.65$. At this point the smooth trend completely vanishes and the data points appear to be randomly distributed in the area. We hypothesize that this critical value of $a$ could be representative of a bifurcation point; however further analysis on the dynamics of these coupled systems interacting would need to be done to investigate this hypothesis further. Additionally we notice that at this critical point, the average carbon decomposed tends to grow significantly. The physical interpretation of this critical point is deeply related to the effectiveness of the competitive rankings as they relate to a biologically diverse system. The results show that beyond a certain point, distributing the biodiversity more closely to the competitive rankings does not have any noticeable effect on the overall effectiveness of the system. \\
This leads to two possibilities beyond the critical point for the system. One possibility is that optimum biodiversity is based on more complex dynamics than the competitive rankings can assess. The other possibility is that the system behaves in a chaotic or random nature beyond this critical point and no clear metric of biodiversity could be correlated to the effectiveness of a given distribution.

\subsection*{Environmental fluctuations}
Realistically, the external environment dictates environmental conditions of the system both in short time scales (year fluctuations for example) and long time scales. Due to the time scales of our model, the effects of very fast fluctuations (due to outside forces) on the scale of days are not considered as relative changes in the model's state have negligible changes on the order of a day. Thus we define our short term fluctuations to be roughly seasonal as this dictates the majority of local environmental variability. To study the effects of these short time scale fluctuations, we choose to implement an abstracted yearly moisture cycle, specifically that of the temperate forest environment. The cycle oscillates between the average yearly maximum and minimum values of $\psi$ twice a year, creating a rough yearly pattern of $\psi$ and thus $\nu$ for a given fungi from experimental data \cite{Zobel2001}\cite{Maynard2019}. Note that this moisture model is limited in assuming one period oscillation between a maximum and a minimum yearly, neglecting the reality of oscillating between local minimum in summer and winter and locate maximums in fall and spring. To incorporate this into the model, $\nu$ and $\psi$ were dynamically allocated over time based on the oscillatory behavior shown in (figure oscillating-psi-nu). Due to time limitations and scarce data on the parameters over these yearly oscillations, only two species of fungi were simulated. A comparison of their respective decomposition contributions with that of similar environmental conditions but static $\nu$ and $\psi$ is shown in (figure oscillating2nonoscillating-psi-contribution). There is no note-worthy new oscillatory motion seen in these results. There is also seen to be a slight increase in the effectiveness of both fungi in the oscillatory case. The interpretation of this increase can be that for time scales on the order of a year, the positives affect of increasing $\nu$ is greater than that of the negative affect of decreasing $\nu$ by the same amount. This would lead us to believe that faster oscillations of $\nu$ will generally have a positive net impact on the decomposition. The frequency limits of this principle would require further investigation. 

\section*{Limitations}
The coupling between our growth model and our decomposition kinetics model was scarcely found in the literature. The literature would typically either adopt a simpler mechanism of fungi growth with an equally complex mechanism of decomposition kinetics or a more complex mechanism for fungi nutrient uptake, but leaving off much of the complexity of the larger environment (especially interactions between fungi). This led to a lack of literature on parameter choices for the coupled system, leaving us to either choose parameter values from models simpler in either regard or to empirically derive our parameters by fitting the model output to other reasonable outputs (as was done with the parameter $G$). As was seen in the model comparison for different methods of coupling the growth equation to the decomposition kinetics, there are some unrealistic disagreements between the early behavior of each growth model; mainly that of a slight increase in the first maximum biomass reached by the carbon limiting growth model. We would expect overall that the biomass gained over time would be less overall when the nutrient level is limited. This is likely due to bad parameter choices, as described above or due to a failure to incorporate a more realistic relationship between the carbon decomposed and the carbon available for the fungi to supplement its own nutrient supply. \\
The Michaelis-Menten dynamics we implemented often considers enzyme concentration to be proportional to the maximum rate of the reaction. Taking this concentration to be proportional to the total biomass in some set region of space does not account for the concentration of enzymes becoming saturated as the fungi spreads out more through space. A more realistic incorporation of this idea into the model would need to address the geometric complexities involved in examining only one region of a much larger fungi growth. An even more realistic treatment of this would account for fungal growth into areas that do not have physical access to the carbon being decomposed. At this point, a simplified one-dimensional growth model would not suffice. \\
The effect of considering biomass proportional to concentration would be seen in the longer term behavior of the model, as the fungi growth becomes limited. We may expect to see more complex dynamics transitioning between the "unbounded" fungi growth and the steady-state behavior of fungi growth. Thus, the largest area of error for our results due to this assumption is likely in the area of time transitioning unbounded growth and steady-state behavior. \\
It was observed in literature that there are direct interactions between fungi growing in the same environment by various mechanisms (NEED CITATION). Our model ignored the effects of these direct interactions and instead focused on indirect interactions via competition for the same nutrients source. The effects of fungi's mechanisms of sabotage via physical means is typically taken to be negligible and the dynamics of such interactions are likely too complicated to consider. However, the effect of space-limitation would likely cause there to be non-negligible limitations on fungal growth due to others obstructing their path or due to total lack of available space. Essentially, our simulation considered fungi competing for food, when in reality, they compete for food and space. In considering these effects, care must be taken to the space of consideration and the limitations of that space. Further review would need to be done to assess the limitations of various spaces, but because we are considering fungi in open ground liter without any obvious obstructions to their growth, spacial limitations on the fungi growth may be considered negligible. This assumption would not apply in the case of geometric isolation of a particular fungi but the geometric access of particular fungi to the substraight is another complexity that would need to be addressed in a higher-dimensional growth model. 

\section*{Limitation pt2}
Values for our parameter, $K_e$ were difficult to find in the literature, as there were not agreed upon dynamics of all four fungi considered. Additionally, cellobiohydrolase was sometimes found to follow different dynamics than the Michaelis-Menten equation \cite{Razavi2015}. To obtain a more accurate value for this parameter, a study of the model's overall sensitivity to varying $K_e$ should be conducted. Additionally, a more accurate representation of fungi interactions would account for more distinctive differences in decomposition dynamics between various enzymes and more than four different enzymes. But any study of alternative decomposition kinetics between enzymes was not done due to time limitations. 
The environmental conditions were analyzed for short term and long term fluctuation (short term being yearly cycles and long term being overall biome transitions over time). Our analysis found that oscillating environmental conditions showed slightly better performance for two species of fungi, but additional studies involving more species of fungi were not studied to to time limitations. In this analysis, the effects of varying $\psi$ only were studied. Realistic environmental fluctuations would see both temperature and $\psi$ fluctuated simultaneously, but for the purpose of studying the general affect of oscillating environmental conditions, this can be taken as a representative sample. The analysis additional left out discussion of fluctuation in between our proposed long term and short term fluctuations. Future studies of the model's response to fluctuating environmental conditions could include a study of the system's response to a continuum of frequencies for environmental fluctuations, as well as studying the effects of fluctuating multiple environmental conditions simultaneously. 
The correlations found between distributions of fungi species and overall performance of the system was found for static environmental conditions in one specific biome (that of an Arid climate). The correlation found before the critical point is poorly understood and would require furhter investigation or numerical fitting to be studied in more detail. The change occurring at the critical point may also be a result of numerical error in approximate solution to the differential equations being solved (although there is no other indication that this is the case). Additionally, Further analysis of this correlation could include assessment of this correlation in different biomes and in oscillating environmental conditions. 

\end{document}